\documentclass{standalone}
\usepackage{graphicx}	
\usepackage{amssymb, amsmath}
\usepackage{color}

\usepackage{tikz}
\usetikzlibrary{math, arrows.meta, decorations.pathreplacing}
\usepackage{pgfmath}
\usepackage{pgfplots}

\definecolor{light}{RGB}{220, 188, 188}
\definecolor{mid}{RGB}{185, 124, 124}
\definecolor{dark}{RGB}{143, 39, 39}
\definecolor{highlight}{RGB}{180, 31, 180}
\definecolor{gray10}{gray}{0.1}
\definecolor{gray20}{gray}{0.2}
\definecolor{gray30}{gray}{0.3}
\definecolor{gray40}{gray}{0.4}
\definecolor{gray60}{gray}{0.6}
\definecolor{gray70}{gray}{0.7}
\definecolor{gray80}{gray}{0.8}
\definecolor{gray90}{gray}{0.9}
\definecolor{gray95}{gray}{0.95}

\pgfplotsset{compat=1.17}
\pgfplotsset{
  colormap={reds}{rgb255=(252, 250, 250) rgb255=(245, 236, 236) rgb255=(235, 218, 218)
                  rgb255=(220, 188, 188) rgb255=(199, 153, 153) rgb255=(185, 124, 124)
                  rgb255=(162, 80, 80) rgb255=(143, 39, 39) rgb255=(124, 0, 0) } 
}

\tikzmath{
  function normal(\x, \m, \s) {
    return exp(-0.5 * (\x - \m) * (\x - \m) / (\s * \s) ) / (2.506628274631001 * \s);
  };
  function indicator(\x, \y) {
    if sqrt( (\x - 1) * (\x - 1) + (\y + 1) * (\y + 1) ) < 0.25 then {
      return 1;
    } else {
      return 0;
    };
  };
}

\begin{document}

\begin{tikzpicture}[scale=1]

  \draw[white] (-1.25, -1) rectangle (7, 5);

  \coordinate (A) at (0, 0.785);
  \coordinate (B) at (0, 4);
  \coordinate (C) at (2.18, 2.475);
  \coordinate (D) at (4.67, 0);  
  \coordinate (E) at (6.85, 1.69); 
  
  \draw[->, >=stealth, line width=1] (A) -- (C);
  \node[above left] at (C) { $x_{2}$ };

  \begin{axis}[xmin=-3, xmax=3, xtick=\empty, 
               ymin=-3, ymax=3, ytick=\empty,
               zmin=0, zmax=0.175, ztick=\empty,
               axis lines=none,              
               line width=1,
               colormap name=reds]
        \addplot3 [domain=0:360, samples=60, samples y=0, dark]
          ( {+1 + 0.75 * (1 + 0.5 * cos(2 * x)) * sin(x)}, 
            {-1 + 0.75 * (1 + 0.5 * cos(2 * x)) * cos(x)}, {0} );
  \end{axis}
  
  \node[dark] at (3.875, 0.825) { $\mathsf{x}$ };
  \draw[dark, line width=1] (3.185, 0.65) -- +(0, 2.55);
  \draw[dark, line width=1] (4.503, 1.0) -- +(0, 0.99);
  
  \begin{axis}[xmin=-3, xmax=3, xtick=\empty, 
               ymin=-3, ymax=3, ytick=\empty,
               zmin=0, zmax=0.175, ztick=\empty,
               axis lines=none,              
               line width=1,
               colormap name=reds]
        \addplot3[surf, shader=interp, samples=70, fill opacity=0.75, domain=-2.9:3] 
          {normal(x, 0, 1) * normal(y, 0, 1)};
        \addplot3 [domain=0:360, samples=60, samples y=0, dark]
          ( {+1 + 0.75 * (1 + 0.5 * cos(2 * x)) * sin(x)}, 
            {-1 + 0.75 * (1 + 0.5 * cos(2 * x)) * cos(x)}, 
            {  normal(+1 + 0.75 * (1 + 0.5 * cos(2 * x)) * sin(x), 0, 1) 
             * normal(-1 + 0.75 * (1 + 0.5 * cos(2 * x)) * cos(x), 0, 1)} );
  \end{axis}
  
  \node[dark] at (3.875, 0.825 + 1.7) { $\pi(\mathsf{x})$ };
  
  \draw[->, >=stealth, line width=1] (A) -- (D);
  \node[below right] at (D) { $x_{1}$ };
  
  \fill (A) circle (0.0175);
  \draw[->, >=stealth, line width=1] (A) -- (B);
  \node[above] at (B) { $\frac{ \mathrm{d} \pi \hphantom{ {}^{2} } }{ \mathrm{d} \lambda^{2} } (x_{1}, x_{2})$ };
  
\end{tikzpicture}

\end{document}  