\documentclass{standalone}
\usepackage{graphicx}	
\usepackage{amssymb, amsmath}
\usepackage{color}

\usepackage{tikz}
\usetikzlibrary{math, arrows.meta, decorations.pathreplacing}
\usepackage{pgfmath}
\usepackage{pgfplots}

\definecolor{light}{RGB}{220, 188, 188}
\definecolor{mid}{RGB}{185, 124, 124}
\definecolor{dark}{RGB}{143, 39, 39}
\definecolor{highlight}{RGB}{180, 31, 180}
\definecolor{gray10}{gray}{0.1}
\definecolor{gray20}{gray}{0.2}
\definecolor{gray30}{gray}{0.3}
\definecolor{gray40}{gray}{0.4}
\definecolor{gray60}{gray}{0.6}
\definecolor{gray70}{gray}{0.7}
\definecolor{gray80}{gray}{0.8}
\definecolor{gray90}{gray}{0.9}
\definecolor{gray95}{gray}{0.95}

\pgfplotsset{compat=1.17}
\pgfplotsset{
  colormap={reds}{rgb255=(252, 250, 250) rgb255=(245, 236, 236) rgb255=(235, 218, 218)
                  rgb255=(220, 188, 188) rgb255=(199, 153, 153) rgb255=(185, 124, 124)
                  rgb255=(162, 80, 80) rgb255=(143, 39, 39) rgb255=(124, 0, 0) } 
}

\tikzmath{
  function normal(\x, \m, \s) {
    return exp(-0.5 * (\x - \m) * (\x - \m) / (\s * \s) ) / (2.506628274631001 * \s);
  };
}

\begin{document}

\begin{tikzpicture}[scale=1]

  \begin{scope}[shift={(0, 0)}]
    \draw[white] (-1, -1) rectangle (7.25, 5);
  
    \coordinate (A) at (0, 0.785);
    \coordinate (B) at (0, 4);
    \coordinate (C) at (2.18, 2.475);
    \coordinate (D) at (4.67, 0);  
    \coordinate (E) at (6.85, 1.69); 
    
    \draw[->, >=stealth, line width=1] (A) -- (C);
    \node[above left] at (C) { $x_{2}$ };
    
    \begin{axis}[xmin=-3, xmax=3, xtick=\empty, 
                 ymin=-3, ymax=3, ytick=\empty,
                 zmin=0, zmax=0.175, ztick=\empty,
                 axis lines=none,              
                 line width=1,
                 colormap name=reds]
          \addplot3[surf, shader=interp, samples=70, fill opacity=0.75, domain=-2.9:3] 
             {normal(x, 0, 1) * normal(y, 0, 1)};
    \end{axis}
    
    \draw[->, >=stealth, line width=1] (A) -- (D);
    \node[below right] at (D) { $x_{1}$ };
    
    \fill (A) circle (0.0175);
    \draw[->, >=stealth, line width=1] (A) -- (B);
    \node[above] at (B) { $p(x_{1}, x_{2})$ };
  \end{scope}
  
  \begin{scope}[shift={(12, 2.25)}]
    \draw[white] (-4.75, -3.25) rectangle (3.5, 2.75);
    
    \foreach \l/\u/\m in {-3.000/-2.750/0.006, -2.750/-2.500/0.009, -2.500/-2.250/0.013, -2.250/-2.000/0.019, -2.000/-1.750/0.026, -1.750/-1.500/0.034, -1.500/-1.250/0.044, -1.250/-1.000/0.053, -1.000/-0.750/0.062, -0.750/-0.500/0.070, -0.500/-0.250/0.076, -0.250/0.000/0.079, 0.000/0.250/0.079, 0.250/0.500/0.076, 0.500/0.750/0.070, 0.750/1.000/0.062, 1.000/1.250/0.053, 1.250/1.500/0.044, 1.500/1.750/0.034, 1.750/2.000/0.026, 2.000/2.250/0.019, 2.250/2.500/0.013, 2.500/2.750/0.009, 2.750/3.000/0.006} {
      \filldraw[fill=mid, draw=dark] (\l, -2) rectangle (\u, {(11 * \m / (\u - \l) - 2)});
    }
    
    \begin{scope}
      \clip (-3, -2) rectangle (3, 2);
      \draw[domain=-3:3, smooth, samples=100, variable=\x, light, line width=1] 
        plot (\x, { 11 * normal(\x, 0, 1.25) - 2});
    \end{scope}
    \node[light] at (2, 0.25) { $f_{*} p(y)$ };
     
    \draw [->, >=stealth, line width=1] (-3.015, -2.00) -- +(0, 4); 
    \draw [->, >=stealth, line width=1] (-3.015, -2.00) -- +(6, 0);
    \node at (0, -2.75) { $y = f(x_{1}, x_{2})$ };
    \node at (-3.75, 0) { $\displaystyle \frac{ f_{*} \pi(\mathsf{b}_{i}) }{ l(\mathsf{b}_{i}) }$ };
    
  \end{scope}
  
\end{tikzpicture}

\end{document}  