\documentclass{standalone}
\usepackage{graphicx}	
\usepackage{amssymb, amsmath}
\usepackage{color}

\usepackage{tikz}
\usetikzlibrary{math, arrows.meta, decorations.pathreplacing}
\usepackage{pgfmath}
\usepackage{pgfplots}

\definecolor{light}{RGB}{220, 188, 188}
\definecolor{mid}{RGB}{185, 124, 124}
\definecolor{dark}{RGB}{143, 39, 39}
\definecolor{highlight}{RGB}{180, 31, 180}
\definecolor{gray10}{gray}{0.1}
\definecolor{gray20}{gray}{0.2}
\definecolor{gray30}{gray}{0.3}
\definecolor{gray40}{gray}{0.4}
\definecolor{gray60}{gray}{0.6}
\definecolor{gray70}{gray}{0.7}
\definecolor{gray80}{gray}{0.8}
\definecolor{gray90}{gray}{0.9}
\definecolor{gray95}{gray}{0.95}

\pgfplotsset{compat=1.17}
\pgfplotsset{
  colormap={reds}{rgb255=(252, 250, 250) rgb255=(245, 236, 236) rgb255=(235, 218, 218)
                  rgb255=(220, 188, 188) rgb255=(199, 153, 153) rgb255=(185, 124, 124)
                  rgb255=(162, 80, 80) rgb255=(143, 39, 39) rgb255=(124, 0, 0) } 
}

\tikzmath{
  function normal(\x, \m, \s) {
    return exp(-0.5 * (\x - \m) * (\x - \m) / (\s * \s) ) / (2.506628274631001 * \s);
  };
    function logistic(\x) {
    if \x > 0 then {
      return 1 / (1 + exp(-\x));
    } else {
      return 1 * exp(\x) / (1 + exp(\x));
    };
  };
  function cdf(\x) {
    return logistic(1.702 * \x);
  };
}

\begin{document}

\begin{tikzpicture}[scale=1]

  \begin{scope}[shift={(0, 0)}]
    \draw[white] (-1, -1) rectangle (7.25, 5);
  
    \coordinate (A) at (0, 0.785);
    \coordinate (B) at (0, 4);
    \coordinate (C) at (2.18, 2.475);
    \coordinate (D) at (4.67, 0);  
    \coordinate (E) at (6.85, 1.69); 
    
    \draw[->, >=stealth, line width=1] (A) -- (C);
    \node[above left] at (C) { $x_{2}$ };
    
    \begin{axis}[xmin=-3, xmax=3, xtick=\empty, 
                 ymin=-3, ymax=3, ytick=\empty,
                 zmin=0, zmax=0.175, ztick=\empty,
                 axis lines=none,              
                 line width=1,
                 colormap name=reds]
          \addplot3[surf, shader=interp, samples=70, fill opacity=0.75, domain=-2.9:3] 
             {normal(x, 0, 1) * normal(y, 0, 1)};
    \end{axis}
    
    \draw[->, >=stealth, line width=1] (A) -- (D);
    \node[below right] at (D) { $x_{1}$ };
    
    \fill (A) circle (0.0175);
    \draw[->, >=stealth, line width=1] (A) -- (B);
    \node[above] at (B) { $p(x_{1}, x_{2})$ };
  \end{scope}
  
  \begin{scope}[shift={(12, 2.25)}]
    \draw[white] (-4.75, -3.25) rectangle (3.5, 2.75);
    
    \foreach [count=\n] \pl/\ql/\pu/\qu in {0.1/-1.602/0.9/1.602, 0.2/-1.052/0.8/1.052, 0.3/-0.656/0.7/0.656, 0.4/-0.317/0.6/0.317} {
      \pgfmathsetmacro{\prop}{25 * (\n - 1)};
      \colorlet{custom}{dark!\prop!light};
      \fill[custom] (\ql, -1) rectangle (\qu, 1);
      \draw[custom, dashed, line width=1] (\ql, {-1 - 0.3 * (Mod(\n, 2))}) -- (\ql, {+1 + 0.35 * (1 - Mod(\n, 2))});
      \node[custom] at (\ql, {(1 - 2 * Mod(\n, 2)) * 1.5}) { $q_{\pl}$ };
      \draw[custom, dashed, line width=1] (\qu, {-1 - 0.3 * (Mod(\n, 2))}) -- (\qu, {+1 + 0.35 * (1 - Mod(\n, 2))});
      \node[custom] at (\qu, {(1 - 2 * Mod(\n, 2)) * 1.5}) { $q_{\pu}$ };
    }
    \draw[dark, line width=1.5] (0, -1) -- (0, 1);
    \draw[dark, dashed, line width=1] (0, -1.3) -- (0, 1);
    \node[dark] at (0, -1.5) { $q_{0.5}$ };
    
    \draw [<->, >=stealth, line width=1] (-3.015, -1.00) -- +(6, 0);
    \node at (0, -2.5) { $y = f(x_{1}, x_{2})$ };
    
  \end{scope}
  
\end{tikzpicture}

\end{document}  