\documentclass{standalone}
\usepackage{graphicx}	
\usepackage{amssymb, amsmath}
\usepackage{color}

\usepackage{tikz}
\usetikzlibrary{math}
\usepackage{pgfmath}
\usepackage{pgfplots}

\definecolor{light}{RGB}{220, 188, 188}
\definecolor{mid}{RGB}{185, 124, 124}
\definecolor{dark}{RGB}{143, 39, 39}
\definecolor{highlight}{RGB}{180, 31, 180}
\definecolor{gray10}{gray}{0.1}
\definecolor{gray20}{gray}{0.2}
\definecolor{gray30}{gray}{0.3}
\definecolor{gray40}{gray}{0.4}
\definecolor{gray50}{gray}{0.5}
\definecolor{gray60}{gray}{0.6}
\definecolor{gray70}{gray}{0.7}
\definecolor{gray80}{gray}{0.8}
\definecolor{gray90}{gray}{0.9}
\definecolor{gray95}{gray}{0.95}

\pgfplotsset{compat=1.17}
\pgfplotsset{
  colormap={reds}{rgb255=(252, 250, 250) rgb255=(245, 236, 236) rgb255=(235, 218, 218)
                  rgb255=(220, 188, 188) rgb255=(199, 153, 153) rgb255=(185, 124, 124)
                  rgb255=(162, 80, 80) rgb255=(143, 39, 39) rgb255=(124, 0, 0) } 
}

\tikzmath{
  function multinormal(\x, \y) {
    return 0.159154943091895 / (1 * 1) * exp( -0.5 * (\x * \x + \y * \y) / (1 * 1) );
  };
  function Y(\x, \n) {
    \a = 0.05 * \x * \x * \x - 0.2 * \x - 0.175 + \n;
    if \a < -2.5 then {
      return -2.5;
    } else {
      return \a;
    };
  };
}

\begin{document}

\begin{tikzpicture}[scale=1]

  \begin{scope}[shift={(0, 0)}]
    \draw[white] (-1, -0.25) rectangle (7.75, 7);

    \pgfmathsetmacro{\theta}{120}
    \pgfmathsetmacro{\phi}{45}
  
    \coordinate (A) at (2.5, 3.275);
    \coordinate (B) at (2.5, 6);
    \coordinate (C) at (6.9, 2.05);
    \coordinate (D) at (0, 1.2);  
    \coordinate (E) at (6.85, 1.69); 
    
    \draw[->, >=stealth, line width=1] (A) -- (D);
    \node[above left] at (D) { $x_{1}$ };
    
    \draw[->, >=stealth, line width=1] (A) -- (C);
    \node[above right] at (C) { $x_{2}$ };
    
    \fill (A) circle (0.0175);
    \draw[->, >=stealth, line width=1] (A) -- (B);
    \node[above] at (B) { $p(x_{1}, x_{2}) \cdot I_{f^{-1}(y)}(x_{1}, x_{2})$ };
    
    \foreach \d in {-2.5, -2.4, ..., 3} {
      \begin{axis}[view={\theta}{\phi},
                   xmin=-2.5, xmax=+2.5, xtick=\empty, 
                   ymin=-2.5, ymax=+2.5, ytick=\empty,
                   zmin=0,    zmax=0.4, ztick=\empty,
                   axis lines=none,              
                   line width=1,
                   colormap name=reds]
        \addplot3 [domain=-2.5:2.5, samples=60, samples y=0, gray60]
          ( {\x}, {Y(\x, \d)}, {0} );
          
        \addplot3 [domain=-2.5:2.5, samples=60, samples y=0, dark]
          ( {\x}, {Y(\x, \d)}, {multinormal(\x, Y(\x, \d))} );
      \end{axis}
    }
    
  \end{scope}
  
\end{tikzpicture}

\end{document}  